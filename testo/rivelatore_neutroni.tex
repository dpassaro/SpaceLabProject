\section{Rivelatore di neutroni}
Essendo particelle neutre, i neutroni non possono essere rivelati con \emph{detectors} che sfruttano il principio della ionizzazione, perciò una strumentazione dedicata è necessaria. I rivelatori di neutroni sono basati sulla rivelazione di prodotti secondari (fotoni, particelle cariche) dalle reazioni nucleari indotte dall'urto del neutrone su elementi tipo $^{10}$B o $^6$Li. Questo sistema di rivelazione è efficace solo per i neutroni termici, e proprio per questo è ideale per la misura sul flusso di neutroni che ci proponiamo di svolgere.  

Il rivelatore di neutroni che impiegheremo è il dosimetro CT007-T della GammaGuard, che accoppia una plastica drogata con $^6$Li a dei SiPM. Le caratteristiche principali sono: \textbf{alimentazione} - due pile AA; \textbf{peso} - 140 g; \textbf{elettronica di lettura} - il rivelatore comunica tramite Bluetooth con una applicazione per Android, dove i dati vengono caricati a una frequenza di 5 Hz e possono essere integrati con le informazioni GPS. Utilizzeremo per questo scopo una scheda Raspberry Pi con sistema Android.  


\section{Rivelatori di particelle cariche}

Tutte le misure di flusso passano attraverso la conoscenza e lo studio dettagliato dell'efficienza dei singoli detector che costituiscono il nostro apparato. Tale studio sarà effettuato a terra presso i laboratori dell'INFN come operazione preliminare al volo, presumibilmente nelle settimane precedenti al lancio. 

La rivelazione della componente carica avviene attraverso due diversi apparati: due sistemi basati su scintillatori e una \emph{micromegas}. Entrambi i detector vengono sviluppati appositamente e gentilmente [offerti?] dall'INFN. La prima classe di detector sfrutta il \emph{meccanismo di scintillazione} per rivelare le particelle cariche che attraversano il materiale emettendo fotoni raccolti da due SiPM (Silicon PhotoMultiplier) incollati sulle pareti laterali. 

[Qualche cosa per vantarci dell'uso dei SiPM (?)]

Questo sistema scintillatore--SiPM viene letto da un'elettronica di front-end sviluppata \emph{ad hoc} dall'INFN . 

[schematica, foto, consumi e altre amenita'???]

La micromegas (micro-mesh gaseous strcture) è una camera a ionizzazione [dettagli sulla nostra]. L'elettronica di lettura prevista viene dalla duplicazione di quella disegnata per gli scintillatori. 


\subsection*{Frame}

Per la realizzazione dell’esperimento è necessario realizzare una struttura di supporto in cui poter fissare tutte le apparecchiature per le rilevazioni. Le linee guida per il progetto sono state essenzialmente 3: 1) Leggerezza della struttura; 2) Resistenza termica; 3) Facilità di realizzazione. Per soddisfare questi 3 punti si è deciso di realizzare una struttura stampata in 3D; il materiale che più si adatta alle richieste è il PETG, un materiale meccanicamente e termicamente resistente e che inoltre è facilmente stampabile. Da esperimenti eseguiti su questo materiale si è visto che con un trattamento termico (“thermal annealing”) in cui si raggiungono i -40°C, il materiale perde solamente il 2-3\% delle caratteristiche meccaniche. Per quanto riguarda la temperatura massima, 80°C sono la temperatura di transizione vetrosa del materiale, oltre la quale il materiale passa a perdere le proprie caratteristiche di rigidezza ed assume un comportamento gommoso. Dai dati di precedenti voli si è visto che le temperature più alte vengono raggiunte solamente localmente dai compenti elettronici, pertanto anche se si dovessero superare i 70°C non sarebbe un problema per la resistenza strutturale dei supporti. In Figura \ref{frame} è riportato il supporto meccanico assemblato con i rivelatori.
\begin{figure}
    
    \includegraphics[width=0.242\textwidth]{definitivo___FORSE2.2.PDF}
    \includegraphics[width=0.242\textwidth]{ESPLOSO2.2.PDF}

    \caption{A sinistra: frame di sostegno con rivelatori montati; a destra: vista esplosa. In obliquo vengono allocati gli scintillatori, mentre al centro viene posizionata la micromegas; sottostante la micromegas, viene posizionato il rivelatore di neutroni ed eventualmente una terza lastra scintillatrice.}
    \label{frame}

\end{figure}
